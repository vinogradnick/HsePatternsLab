\documentclass[PI,LAB]{HSEUniversity}
% Возможные опции: KR или VKR; PI или BI
\usepackage{svg}
\usepackage{listings}

\usepackage{plantuml}
\title{Оценка архитектуры программной системы для заданной предметной области}
\author{Виноградов Никита Андреевич}
\supervisor{к.т.н., доцент кафедры Информационных технологий в бизнесе НИУ ВШЭ-Пермь}{А.В.~Кычкин}

\Year{2020}


% Ссылка на файл с описание библиографии
\bibliography{library.bib}

%%%%%%%%%%%%%%%%%%%%%%%%%%%%%%%%
%%% ТЕКСТ РАБОТЫ %%%%%%%%%%%%%%%
\begin{document}

    % Обязательные элементы оформления: заголовочный слайд, аннотация, оглавление
    \maketitle



\chapter{Оценка архитектуры программной системы для заданной предметной области. }
\section {Описание проекта}
\emph{Описание проекта} - Автоматизированная система нагрузочного тестирования для веб-серверов (облачных систем).

\emph{Цель проекта} - Автоматизировать и упростить процесс тестирования нагрузки и последующего исследования показателей для принятия решений.

\emph{Задачи проекта} -

\emph{Области применения} - Облачные  распределенныые системы,  ETL и MapReduce системы.

\emph{Составные части}:\begin{enumerate}  
	\item Сервер системы - выполняет роль единой точки обращения к системе, позволяет назначать работы для воркеров и управленяет работой Воркеров.
	\item Воркеры - рабочие элементы системы выполняющие роль тестировщиков, в данной системе  они могут выполнять 2 вида работ:
	\begin{enumerate}
		\item Hit-Based тестирование.
		\item Сценарное тестирование.
	\end{enumerate} 
	\item Мониторинг - используется для просмотра информации о проведенных нагрузочных тестах, составлении графиков зависимостей.
	\item Сборщик логов - используется для сбора информации со всех элементов системы и записи процесса выполнения на диск, для последующей отладки или просмотра.
	\item Агенты - помощники основной системы, в основной части выполняются на target машине для которой выполняется тестирование чтобы отследить данные по процессору, памяти, дисковой нагрузке, также агенты  используется для предотвращения падения системы в случае высокой нагрузки сервера нагрузочного тестирования.
	\item Планировщик - используется для организации работ, которые выполняют воркеры.В текущей системе распределяет потоки выполнения для увеличения общей вычислительной мощности.
	\item Веб-Сервер - используется для загрузки конфигурации нагрузочных тестов и просмотра информации по возможностям системы и свободным воркерам.
	\item CLI-клиент - клиент системы выполняющий функции веб-сервера, только в режиме терминала.
\end{enumerate}


	



\section {Компоненты системы}
\begin{enumerate}
	\item СУБД - для данной системы была выбрана колоночная Time-Series База данных ClickHouse. Данное решение обусловлено  возможностями системы по предоставлению нагрузки и генерации метрик в ходе тестирования, (Было проведено  тестирование возможностей системы и при режиме тестирования \emph{максимальной производительности}, система может выдавать 1.8 Млн RPS) в ходе этого необходимо эффективно доставлять метрики в базу данных а не копить в памяти сервера.
	\item Jenkins X Api Client - инструмент для подключения к  системе сборки и доставки приложений.
	\item  Gitlab CI/CD  Api Client - инструмент для подключения к  системе сборки и доставки приложений.
	\item Kubernetes Api Client - инструмент для подключения к системе оркестрации контейнеров, будет использовать для проверки масштабирования компонентов облачных систем.
	\item Webhooks -инструменты создания конечных точек в системе на сервере, для выполнения определенных команд.
	\item gRPC - инструмент эффективного подключения сервисов внутри и между дата-центрами с помощью подключаемой поддержки для балансировки нагрузки, трассировки, проверки работоспособности и аутентификации
	\item TabiX - инструмент для анализа данных из ClickHouse и построения графиков и диаграмм по тестируемым системам
	\end{enumerate}

\section {Архитектурный стиль}
В данной системе используется несколько подходов системы
\begin{enumerate}
	\item Клиент-Сервер - используется для взязи CLI CLient и Воркеров в сервером системы.
	 \item Master-Slave - используется для масштабирования воркеров, в задачах тестирования серверов с большими возможностями.
	 \item Observer - используется для сбора логов и метрик в компонентах системы.
\end{enumerate}



	\begin{plantuml}
		@startuml
		class Car
		
		Driver - Car : drives >
		Car *- Wheel : have 4 >
		Car -- Person : < owns
		@enduml
	\end{plantuml}

\section {Способ применения архитектурных решений}
   % \begin{FIGURE}[h]{Структура классов паттерна строитель\label{fig:example-figure}}
   %    \includegraphics[width=0.2\textwidth]{../out/diagrams/builder/builder_default_class}
   % \end{FIGURE}
\chapter{Диаграмма вариантов использования проектируемой системы в соответствии с нотацией UML}
\chapter{Анализ архитектуры с использованием SAAM}

 %\begin{FIGURE}[h]{Диаграмма последовательности паттерна строитель\label{fig:example-figure}}
%	\includegraphics[width=0.5\textwidth]{../out/diagrams/builder/builder_default_sq}
%\end{FIGURE}
    

\end{document}
